\chapter[Atividades e Cronograma]{Atividades e Cronograma}

\section{Cronograma de Execução}

% ######## init table ########
\begin{table}[h]
 \centering
% distancia entre a linha e o texto
 {\renewcommand\arraystretch{1.25}
 \caption{Cronograma de etapas}
 \begin{tabular}{ l l }
  \cline{1-1}\cline{2-2}  
    \multicolumn{1}{|p{3.850cm}|}{Período \centering } &
    \multicolumn{1}{p{4.217cm}|}{Atividade \centering }
  \\  
  \cline{1-1}\cline{2-2}  
    \multicolumn{1}{|p{3.850cm}|}{01/07 - 19/07 \centering } &
    \multicolumn{1}{p{4.217cm}|}{Etapa 01 \centering }
  \\  
  \cline{1-1}\cline{2-2}  
    \multicolumn{1}{|p{3.850cm}|}{20/07 - 19/08 \centering } &
    \multicolumn{1}{p{4.217cm}|}{Etapa 02 \centering }
  \\  
  \cline{1-1}\cline{2-2}  
    \multicolumn{1}{|p{3.850cm}|}{20/08 - 19/09 \centering } &
    \multicolumn{1}{p{4.217cm}|}{Etapa 03 \centering }
  \\  
  \hline

 \end{tabular} }
\end{table}

ETAPA 01 – Realização  de pesquisa sobre o devops aplicado no contexto de softwares livres. Nesta etapa foram realizados estudos na literatura a respeito de diversos temas relacionados à pesquisa, como por exemplo: levantamento de principais formas de aplicação de devops, estado da arte dos pipelines devops, compilação de fontes e artigos relevantes.

ETAPA 02 – Estudo e aplicação das técnicas de DevOps nos softwares culturais em desenvolvimento.

ETAPA 03 – Elaboração de relatório final com dados sobre as melhores formas de se aplicar o pipeline devops
