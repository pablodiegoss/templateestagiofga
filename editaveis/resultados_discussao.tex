\chapter[Resultados e Discussão]{Resultados e Discussão}

Ao longo do trabalho os conceitos vistos no curso de Engenharia de Software foram trabalhados e auxiliaram nas tarefas. O conhecimento das técnicas de programação, juntamente com medição, análise e testes de software permitiu que a entrega contínua dos softwares fosse planejada e executada de forma controlada e mantendo os padrões necessários para o cliente.

O estudo do estado da arte do DevOps permitiu o desenvolvimento de pipelines para entrega contínua em diversos ambientes simultâneos, como desenvolvimento, homologação e produção, além do empacotamento e releases destas diferentes versões do software. Todos as ações desse tipo se realizam de forma automatizada, foi disseminada entre a equipe e documentada para os projetos futuros.

O trabalho em conjunto com o Ministério da Cultura, pode ser considerado um sucesso e trouxe bons resultados profissionais e  possibilitou a produção de um software que possa ser utilizado em prol do incentivo cultural do país.
